\documentclass{article}

% AMS math symbols
\usepackage{amsfonts}
\usepackage{amsmath}
\usepackage{amssymb}

% Evolutional algorithms journal template
\usepackage{ecj}

% set spacing in enums and items
\usepackage{enumitem}

% eps figures
\usepackage{epsfig}

% boxed figures
\usepackage{float}

% reduce margins
\usepackage{fullpage}

% citations with author
\usepackage{natbib}

\newcommand{\concat}{\ensuremath{+\!\!\!\!+\,}}

\floatstyle{boxed} 
\restylefloat{figure}

\setlength{\parskip}{7pt}
\setlength{\parsep}{0pt}
\setlength{\parindent}{0pt}

% ADDING A FIGURE
%\begin{figure}[t]
%\begin{center}
%\psfig{file=Bsimilar.eps,width=300pt}
%\end{center}
%\caption{NKY algorithm: performance of different implementations.}
%\label{fig:Bsimilar}
%\end{figure}
%

% CITING
% \citep{citation}

\begin{document}

% uncomment this to get journal footers
%\ecjHeader{x}{x}{xxx-xxx}{200X}{genetic vs deterministic algorithms for LCS}{C. A. Andreoni, U. Masood}
\title{Caching Fractal B+ Trees}

\author{\name{\bf Claudio Alberto Andreoni} \hfill \addr{caa@mit.edu}\\ 
        \addr{Department of Mathematics, Massachusetts Institute of Technology, 
        Cambridge, 02139, United States}
\AND
       \name{\bf Usman Masood} \hfill \addr{usmanm@mit.edu}\\
        \addr{Department of Electrical Engineering and Computer Science, Massachusetts Institute of Technology, 
        Cambridge, 02139, United States}
}

\maketitle

\begin{abstract}


Genetic algorithms (GAs) have been proposed for the solution of the Longest Common Subsequence problem (LCS).
Their ability to perform efficiently on practical use cases has been analyzed both theoretically and empirically in the literature.
Most benchmarks focused on the NP-complete case of finding the LCS of more than two sequences,
obtaining strongly positive performance outcomes.

In this paper, we explore the time performance of a GA algorithm for the P-complete LCS problem on two sequences
with the purpose of contrasting it to that of polynomial deterministic algorithms.
We produce an improved implementation of the algorithm, along with two other deterministic algorithms, and we compare their
performance under real-world test cases.
We conclude that the GA algorithm fares mostly poorly compared to deterministic algorithms.
\end{abstract}

\begin{keywords}
Genetic algorithms,
genetic heuristics,
dynamic programming,
performance comparison,
longest common subsequence.
\end{keywords}

\section{Introduction}
For all benchmarks in this paper, we use a system with the following specifications:

\begin{tabular}{l|l|c}
	Processor & Model & Intel Core i7 920 \\
	& Clock freq. & 4.0 GHz \\
	& Phys. cores & 4 \\
\hline
	RAM & Capacity & 6 GB \\
	& Clock freq. & 1333 MHz \\
	& Channels & 3
\end{tabular}

All code was written in the C language, and compiled with gcc version 4.6.1

\small

\bibliographystyle{apalike}
\bibliography{final}


\end{document}
